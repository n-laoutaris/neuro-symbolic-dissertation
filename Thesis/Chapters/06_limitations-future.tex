\chapter{Limitations \& Future Work}

% Maybe we should structure this by theme, and each theme will contain both its limitation and its proposal for future work, instead of breaking the chapter into 2.

\section{Limitations}
\label{subsec:api_limits}
limited document sample size, API restrictions and the reliance on a specific vendor's ecosystem, limited prompt engineering, limited models (free tier). 

Replication Crisis: 
Critique the reliance on proprietary Model-as-a-Service infrastructure, arguing that operational instability renders them unsuitable for critical pipelines.
Maybe this can be in the Limitations chapter.

more human interpretation of the results is needed. it's difficult to pinpoint logic errors within complex queries. 

\section{Future Research Directions}
a roadmap for using local open-source models to ensure sovereignty, the implementation of iterative "Self-Correction" agents to fix syntax errors, ideas for more robust testing of this kind of pipeline, ideas not implemented by this work for scoping reasons.

The decision to base the schema on an existing vocabulary was with good reason. This design allows the graphs generated by the pipeline to include more classes of the used ontologies, for future integration with more sophisticated systems. The pipeline itself could also be expanded upon to include more classes.

Make an effort to find why the pipeline failed logically when it did. Many times it was the preconditions extraction and not the text-to-logic part. many times the models failed to correctly interpret what was a precondition and what was an administrative step.

Semantic Stability: data from semantic similarity metrics drawn from past run artifacts on file.

future iterations of the pipeline must prioritize Precision (Trustworthiness) over Recall (Coverage), potentially by calibrating the validation logic to be "conservative by default" or by implementing a "Human-in-the-Loop" review for all positive recommendations.

For public services involving means-testing, complex family unit aggregations, or temporal operations, the current neuro-symbolic approach requires either significantly more advanced prompting strategies or even a fundamental shift to deterministic calculation engines for the mathematical components.