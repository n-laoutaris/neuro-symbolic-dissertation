\chapter*{Abstract}
\addcontentsline{toc}{chapter}{Abstract} % Ensure it appears in ToC

Public administration often seems like a maze of bureaucracy, which leaves citizens unaware of public services and benefits to which they are legally entitled. This phenomenon is rooted in the complex nature of legislative requirements and the fragmentation of information sources. As a result, the failure in the proactive delivery of administrative justice is perpetuated. This dissertation addresses this challenge by proposing a \textit{Neuro-Symbolic pipeline} designed to power a \textit{Public Service Recommender System} as an attempt to link unstructured knowledge locked away in legislative texts with personalized service delivery. What begins as an effort to empower citizen awareness, evolves into a generalized engineering framework that prioritizes certainty and interoperability.

The proposed architecture uses LLMs to extract eligibility preconditions from natural language documents and structure them into a symbolic layer of interoperable semantic graphs and executable SHACL/SPARQL validation logic. The system is evaluated through experimental runs, using a mutation testing framework that stress-tests the synthesized logic against heterogeneous citizen scenarios. The results show models achieving up to 80\% syntactic validity, however, functional logic accuracy collapsed to $\approx$25\% in complex use cases. To preserve auditability and compliance, this work argues for \textit{local} (privacy-preserving) model deployments combined with deterministic validation, as results show current LLMs require them to reach acceptable operational trustworthiness.

This dissertation was written as part of the MSc in Data Science at the International Hellenic University. I would like to express my appreciation to my supervisor, Associate Professor Vassilios Peristeras, for his academic support, and also to PhD Candidate Ioannis Konstantinidis for our extensive discussions regarding the technical and conceptual dimensions of this study. Their foundational research and previous work served as inspiration for me to pursue this topic. Finally, I acknowledge the School of Science and Technology and the University Teaching Staff for giving me the opportunity to compose this dissertation.

\begin{flushright}
Nikolaos Laoutaris \\
24 January 2026
\end{flushright}