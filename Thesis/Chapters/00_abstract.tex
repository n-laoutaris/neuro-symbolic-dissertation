\chapter*{Abstract}
\addcontentsline{toc}{chapter}{Abstract} % Ensure it appears in ToC

Public administration often seems like a maze of bureaucracy, with citizens unaware of public services and benefits to which they are legally entitled. This accessibility gap, rooted in complex legislative requirements and fragmented information sources, constitutes a failure in the proactive delivery of administrative justice. This dissertation addresses this challenge by proposing a \textit{Neuro-Symbolic pipeline} designed to power a \textit{Public Service Recommender System} as an attempt to bridge the gap between unstructured knowledge locked away in legislative texts and personalized service delivery. What begins as an effort to empower citizens with awareness, evolves into a generalized engineering framework that ensures certainty and interoperability.

The proposed architecture uses LLMs to extract eligibility preconditions from natural language documents and synthesize them into a structured symbolic layer, consisting of interoperable semantic graphs and executable SHACL validation logic. The system was evaluated through experimental runs, using a mutation testing framework that stress-tests the synthesized logic against heterogeneous citizen scenarios. Results reveal that while models achieved up to 80\% syntactic validity, functional logic accuracy collapsed to $\approx$25\% in complex cases. The findings identify limitations in current LLMs and demonstrate that beyond linguistic fluency, true citizen empowerment necessitates the development of "Sovereign AI" and symbolic validation to maintain explainability and trust.

This dissertation was written as a part of the MSc in Data Science at the International Hellenic University. I would like to express my gratitude to my supervisor, Associate Professor Vassilios Peristeras, for his academic support, and also to PhD Candidate Ioannis Konstantinidis for our extensive discussions regarding the technical and conceptual dimensions of this study. Their foundational research and previous work served as inspiration for me to pursue this topic. Finally, I acknowledge the School of Science and Technology and the University Teaching Staff for providing the support and resources necessary for me to compose this dissertation.

\begin{flushright}
Nikolaos Laoutaris \\
24 January 2026
\end{flushright}