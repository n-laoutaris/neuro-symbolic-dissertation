\chapter*{Abstract}
\addcontentsline{toc}{chapter}{Abstract} % Ensure it appears in ToC

This dissertation was written as a part of the MSc in Data Science at the International Hellenic University.

Public administration often seems like a maze of bureaucracy, with citizens displaying lack of awareness regarding their personal entitlements. This is a direct result of complex legislative requirements and fragmented information sources. This "accessibility gap" renders many citizens unaware of the public services and social benefits to which they are legally entitled, an important failure in the proactive delivery of administrative justice. This dissertation addresses this challenge by proposing a \textit{Neuro-Symbolic pipeline} designed to power a \textit{Public Service Recommender System}, in an attempt to bridge the gap between unstructured knowledge locked away in legislative texts and personalized service. What begins as an effort to empower citizens with awareness, evolves into a generalized engineering framework that ensures this empowerment is grounded in certainty and interoperability. 

The proposed architecture consists of a workflow that grounds neural LLM interpretations in formal semantic standards such as the EU's Core Public Service Vocabulary Application Profile (CPSV-AP). To ensure that this "digital guide" is reliable, the system was evaluated through experimental runs, implementing a deterministic mutation testing framework that stress-tests the synthesized logic against heterogeneous citizen scenarios. The results reveal that while models achieved up to 80\% syntactic validity, functional logic accuracy collapsed to $\approx$25\% in complex cases. These findings identify limitations in current LLMs and demonstrate that true citizen empowerment requires more than linguistic fluency, but rather that it necessitates the development of "Sovereign AI" and symbolic validation to ensure explainability and maintain public trust.

I would like to express my gratitude to my supervising professor, Associate Professor Vassilios Peristeras, for his invaluable guidance and academic support, and also to PhD Candidate Ioannis Konstantinidis for our extensive and fruitful discussions regarding both the technical and conceptual dimensions of this study. It was their foundational research and previous work that served as the primary inspiration for me to pursue this specific topic. Finally, I would like to acknowledge the School of Science and Technology and the rest of the Academic Teaching Staff at the International Hellenic University, for providing the support and the resources necessary for me to be able to compose this dissertation.

\begin{flushright}
Nikolaos Laoutaris \\
24 January 2026
\end{flushright}