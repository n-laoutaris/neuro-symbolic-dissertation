\chapter{Glossary of Terms} \label{app:glossary}
This glossary provides definitions for the terms utilized throughout this dissertation. The terms are categorized by their domain of origin.

\section*{Artificial Intelligence \& Large Language Models}

\begin{description}[style=nextline, leftmargin=1cm, font=\bfseries]

    \item[Neuro-Symbolic Artificial Intelligence]    
    A class of AI approaches that combine neural models, such as deep learning or large language models, with symbolic representations and reasoning mechanisms. Neuro-Symbolic systems aim to integrate the flexibility of neural learning with the formal rigor and interpretability of symbolic logic.

    \item[Mutation Testing]
    An evaluation methodology adapted from software testing in which controlled perturbations (mutations) are applied to structured citizen data in order to assess whether executable eligibility logic correctly accepts or rejects modified profiles.

    
    \item[In-Context Learning (ICL)] 
    The ability of an LLM to perform a task after being shown a few examples within the prompt, without any permanent updates to its underlying neural weights.
    
    \item[Chain-of-Thought (CoT) Prompting] 
    A prompting technique that encourages an LLM to generate intermediate reasoning steps before arriving at a final answer, often used to improve performance on complex logical or mathematical tasks.

    \item[Temperature] 
    A hyperparameter in LLM generation that controls the randomness of the output. A temperature of 0 aims for the most probable/deterministic response, while higher values increase creativity and variability.

    \item[Model-as-a-Service (MaaS)] 
    A cloud-computing model where pre-trained Large Language Models are hosted by a provider (e.g., Google, OpenAI) and accessed by developers via an API, removing the need for local hardware at the cost of dependency on the provider's infrastructure.
    
    \item[Hallucination] 
    A phenomenon in Large Language Models where the system generates text that is syntactically correct but factually incorrect or logically inconsistent with the input context.

\end{description}

\section*{Semantic Web \& Knowledge Engineering}

\begin{description}[style=nextline, leftmargin=1cm, font=\bfseries]

    \item[Knowledge Graph (KG)] 
    A structured representation of information using a graph-based data model, where nodes represent entities and edges represent the relationships between them, grounded in a formal ontology.

    \item[RDFS (Resource Description Framework Schema)] 
    A formal way of representing properties, and the relationships between those properties, within a specific domain of interest.

    \item[Turtle (Terse RDF Triple Language)] 
    A syntax and file format for the Resource Description Framework (RDF) that is designed to be both machine-readable and easily edited by humans.
    
    \item[SHACL (Shapes Constraint Language)] 
    A World Wide Web Consortium (W3C) standard for validating RDF graphs against a set of conditions.
    
    \item[SPARQL (SPARQL Protocol And RDF Query Language)] 
    An RDF query language and data access protocol used to retrieve and manipulate data stored in Resource Description Framework (RDF) format.  

    \item[Graph Edit Distance (GED)] 
    A metric for measuring the similarity between two graphs by calculating the minimum number of operations (node/edge insertions, deletions, or substitutions) required to transform one graph into the other.

    \item[Abstract Syntax Tree (AST)] 
    A tree representation of the abstract syntactic structure of source code (like SPARQL). Comparing ASTs allows for measuring logical similarity rather than just textual overlap.

\end{description}

\section*{Administrative Law \& Public Governance}

\begin{description}[style=nextline, leftmargin=1cm, font=\bfseries]

    \item[Public Service Recommender System]
    A class of digital governance systems that automatically assess citizen eligibility for public services by continuously evaluating available administrative data, proactively notifying users of potential entitlements rather than requiring manual discovery.

    \item[Once-Only Principle (OOP)] 
    A European Union e-government strategy aimed at ensuring that citizens and businesses only have to provide standard information to administrations once, with authorities then sharing this data internally.

    \item[Single Digital Gateway (SDG)]
    A European Union initiative that creates a single point of access to information, administrative procedures and assistance services across borders, aiming to reduce barriers for citizens and businesses within the Single Market.
    
    \item[Digital Sovereignty] 
    The capacity of a state to exert authority over its digital infrastructure, including the "code" of its laws, without being dependent on proprietary, foreign-controlled ecosystems.

\end{description}